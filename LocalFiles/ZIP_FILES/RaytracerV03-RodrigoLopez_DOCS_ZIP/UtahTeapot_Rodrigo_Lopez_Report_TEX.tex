\documentclass[12pt, letterpaper]{article}

\usepackage{geometry}
\usepackage{ragged2e}
\usepackage{setspace}
\usepackage{newtxtext}

\geometry{
    top=2em,
    bottom=2em,
    left=5em,
    right=5em
}

\title{The Utah Teapot: A Icon of Computer Graphics}
\author{Rodrigo Lopez \\ ID:0262146}

\begin{document}

    \maketitle
    
    \onehalfspacing
    \justifying

    \noindent \textit{"The teapot is the ‘hello world’ of computer graphics." – Martin Newell.}
    
    \setlength{\parindent}{2em}
    A humble teapot, originally modeled in 1975 by computer graphics pioneer Martin Newell at the University of Utah, has become one of the most recognizable symbols in the history of 3D rendering. Known as the \textit{Utah Teapot}, this unassuming object was crafted from a simple physical teapot Newell had in his lab. Its smooth curves, handle, spout, and lid made it an ideal test subject for experimenting with shading algorithms, Bézier surfaces, and texture mapping. Little did Newell know that his model would become a cornerstone of computer graphics research.

    \setlength{\parindent}{2em}
    The teapot’s mathematical representation consists of just 32 Bézier patches, a revolutionary technique at the time. Its simplicity and elegance allowed researchers to focus on developing rendering techniques rather than complex modeling. Early adopters like Jim Blinn and Edwin Catmull used it to test innovations such as environment mapping, ray tracing, and anti-aliasing. The teapot’s ubiquity in academic papers and demos earned it the nickname \textit{the "Stanford Bunny" of its era}, though its cultural impact far surpassed that of later test models.

    \setlength{\parindent}{2em}
    Beyond academia, the Utah Teapot infiltrated pop culture. It appeared in Pixar’s \textit{Toy Story} (1995) as an Easter egg, in \textit{3D Studio Max} as a default primitive, and even in the \textit{Simpsons} as a gag about computer-generated imagery. Its symbolic status was cemented when the \textit{Computer History Museum} in Mountain View, California, acquired the original physical teapot in 2014, displaying it as an artifact of technological progress.

    \setlength{\parindent}{2em}
    Technically, the teapot’s legacy lies in its role as a benchmark. It helped standardize testing for rendering engines, GPU performance, and even AI-based image generation. Modern tools like Blender and OpenGL still include it as a built-in primitive. Its influence extends to virtual reality and real-time rendering, where it remains a reference for lighting and shadow accuracy.

    \setlength{\parindent}{2em}
    The Utah Teapot exemplifies how a simple idea can shape an entire field. From its origins as a practical tool for researchers to its status as a cultural icon, it bridges the gap between technical innovation and artistic expression. Today, it stands not just as a test object but as a tribute to the creativity and curiosity that drive computer graphics forward.

\end{document}